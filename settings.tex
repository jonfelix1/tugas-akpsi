%-----------------------------------------------------------------------------%
% Informasi Mengenai Dokumen
%-----------------------------------------------------------------------------%
% 
% Judul laporan. 
\var{\judul}{Network Automation}
% 
% Tulis kembali judul laporan, kali ini akan diubah menjadi huruf kapital
\Var{\Judul}{NETWORK AUTOMATION}
% 
% Tulis kembali judul laporan namun dengan bahasa Ingris
\var{\judulInggris}{Network Automation}

% 
% Tipe laporan, dapat berisi Skripsi, Tugas Akhir, Thesis, atau Disertasi
\var{\type}{Tugas AKPSI}
% 
% Tulis kembali tipe laporan, kali ini akan diubah menjadi huruf kapital
\Var{\Type}{Tugas AKPSI}
% 
% Tulis nama penulis 
\var{\penulis}{Nama Penulis}
% 
% Tulis kembali nama penulis, kali ini akan diubah menjadi huruf kapital
\Var{\Penulis}{Nama Penulis}
% 
% Tulis NPM penulis
\var{\npm}{NPM}
% 
% Tuliskan Fakultas dimana penulis berada
\Var{\Fakultas}{Ilmu Komputer}
\var{\fakultas}{Ilmu Komputer}
% 
% Tuliskan Program Studi yang diambil penulis
\Var{\Program}{MAGISTER TEKNOLOGI INFORMASI}
\var{\program}{Magister Teknologi Informasi}
% 
% Tuliskan tahun publikasi laporan
\Var{\bulan}{Desember}
\Var{\tahun}{2024}
% 
% Tuliskan gelar yang akan diperoleh dengan menyerahkan laporan ini
\var{\gelar}{Magister Teknologi Informasi}
% 
% Tuliskan tanggal pengesahan laporan, waktu dimana laporan diserahkan ke 
% penguji/sekretariat
\var{\tanggalPengesahan}{XX Juli 2019} 
% 
% Tuliskan tanggal keputusan sidang dikeluarkan dan penulis dinyatakan 
% lulus/tidak lulus
\var{\tanggalLulus}{XX Juli 2019}
% 
% Tuliskan pembimbing 
\var{\pembimbing}{Prof. ???}
% 
% Alias untuk memudahkan alur penulisan paa saat menulis laporan
\var{\saya}{Penulis}

%-----------------------------------------------------------------------------%
% Judul Setiap Bab
%-----------------------------------------------------------------------------%
% 
% Berikut ada judul-judul setiap bab. 
% Silahkan diubah sesuai dengan kebutuhan. 
% 
\Var{\kataPengantar}{Kata Pengantar}
\Var{\babSatu}{BUSINESS REQUIREMENT}
\Var{\babDua}{FEASIBILITY ANALISIS}
\Var{\babTiga}{PROJECT PLAN}
\Var{\babEmpat}{REQUIREMENTS DEFINITION}
\Var{\babLima}{REQUIREMENT ANALYSIS STRATEGY}
\Var{\babEnam}{FUNCTIONAL MODELS}
\Var{\babTujuh}{STRUCTURAL MODELS}
\Var{\babDelapan}{BEHAVIORAL MODELS}
\Var{\babSembilan}{USER INTERFACE LAYER DESIGN}
\Var{\babSepuluh}{PHYSICAL ARCHITECTURE LAYER DESIGN}
\Var{\babSebelas}{WORKING SYSTEM PROTOTYPE}
\Var{\babDuabelas}{TESTING}
\Var{\babTigabelas}{INSTALLATION AND OPERATIONS}

\Var{\kesimpulan}{Kesimpulan dan Saran}
