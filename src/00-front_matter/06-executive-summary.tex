%
% Halaman Abstrak
%
% @author  Andreas Febrian
% @version 1.00
%

\chapter*{Executive Summary}

\vspace*{0.2cm}
{
% 	\setlength{\parindent}{0pt}
	
% 	\begin{tabular}{@{}l l p{10cm}}
% 		Nama&: & \penulis \\
% 		Program Studi&: & \program \\
% 		Judul&: & \judul \\
% 		Pembimbing&: & \pembimbing \\
% 	\end{tabular}

% 	\bigskip
% 	\bigskip

PT. XYZ saat ini menghadapi tantangan dalam pengelolaan jaringan yang melibatkan ratusan perangkat dengan kebutuhan konfigurasi yang berulang. Masalah utama muncul dari proses manual yang memerlukan banyak tenaga kerja dan waktu. Setiap perubahan konfigurasi jaringan harus dilakukan satu per satu, sehingga meningkatkan risiko kesalahan manusia (human error) dan memperlambat waktu penyelesaian. Selain itu, seiring dengan bertambahnya perangkat dan kompleksitas jaringan, PT. XYZ mengalami kendala dalam efisiensi operasional, yang akhirnya berdampak pada produktivitas tim dan kualitas layanan.

Untuk mengatasi masalah ini, PT. XYZ sedang mempertimbangkan implementasi Network Automation, yang diharapkan dapat menyederhanakan proses pengelolaan jaringan. Dengan otomatisasi, konfigurasi, monitoring, dan provisioning dapat dilakukan secara terpusat melalui dashboard tunggal. Hal ini akan mengurangi kebutuhan akan intervensi manual yang signifikan, mempercepat penyelesaian tugas, dan mengurangi potensi kesalahan.

Implementasi sistem network automation juga membawa berbagai keuntungan bisnis, baik yang terukur (tangible) maupun tidak terukur (intangible). Di antaranya adalah pengurangan biaya operasional karena berkurangnya intervensi manual, peningkatan efisiensi jaringan, pengurangan downtime, serta peningkatan skala operasional tanpa perlu menambah personel teknis. Dari sisi intangible, otomatisasi akan meningkatkan kualitas layanan, mempercepat inovasi, meningkatkan kepuasan pengguna, dan memperkuat ketahanan jaringan terhadap gangguan.

Namun, PT. XYZ juga menghadapi beberapa tantangan dalam implementasi ini. Kompleksitas infrastruktur jaringan yang terdiri dari berbagai vendor dan teknologi berbeda memerlukan solusi yang mendukung multi-vendor dan protokol terbuka. Selain itu, tim jaringan memerlukan peningkatan keterampilan dalam scripting, pemrograman, dan penggunaan API untuk mendukung transisi ke network automation.

Secara keseluruhan, otomatisasi jaringan di PT. XYZ diharapkan dapat meningkatkan efisiensi operasional, mengurangi biaya, dan memberikan layanan yang lebih stabil dan responsif, serta menyiapkan perusahaan untuk menghadapi pertumbuhan dan ekspansi di masa depan.

	\bigskip

% 	Kata kunci:\\
% 	Informasi, \textit{information literacy}, \textit{information skills}
}

\newpage