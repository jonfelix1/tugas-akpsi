%-----------------------------------------------------------------------------%
\addChapter{Lampiran 1}
\chapter*{Lampiran 1}

\begin{table}[h!]
    \centering
    \renewcommand{\arraystretch}{1.3}
    \begin{tabular}{|p{4cm}|p{10cm}|}
        \hline
        \textbf{Narasumber} & Haykal Rasidi \\
        \hline
        \textbf{Jabatan} & IT Infrastructure Manager PT. XYZ \\
        \hline
        \textbf{Tanggal} & 17 Agustus 2024 \\
        \hline
        \textbf{Kode} & HR: Haykal Rasidi \newline TR: Tio Ramadhan \\
        \hline
    \end{tabular}
\end{table}

\begin{longtable}{|p{1cm}|p{1.5cm}|p{11cm}|}
    \hline
    \textbf{No} & \textbf{Kode} & \textbf{Pertanyaan/Jawaban} \\
    \hline
    1 & TR & Apa saja tantangan utama yang PT. XYZ hadapi dalam pengelolaan jaringan saat ini? \\
    \hline
    & HR & Tantangan utama adalah memastikan perangkat network bekerja 24x7 dengan baik. Dikarenakan bisnis kami memerlukan high availability network yang tinggi. Managed ratusan network device menjadi tantangan tersendiri. \\
    \hline
    2 & TR & Seberapa sering perubahan konfigurasi atau penambahan perangkat dilakukan di jaringan PT. XYZ? \\
    \hline
    & HR & Perubahan minor hampir setiap hari ada terkait kebutuhan user, namun untuk major changes bergantung dengan kebutuhan. \\
    \hline
    3 & TR & Apakah PT. XYZ merasa proses manual dalam mengelola jaringan saat ini memakan terlalu banyak waktu atau rentan terhadap kesalahan? \\
    \hline
    & HR & Pengelolaan manual memang memiliki tantangan sendiri, di mana kita manage ratusan perangkat network dan ada celah dan rentan kesalahan konfigurasi dan deployment. \\
    \hline
    4 & TR & Berapa banyak waktu yang biasanya dihabiskan oleh Tim Bapak untuk tugas-tugas berulang seperti provisioning, monitoring, dan troubleshooting jaringan? \\
    \hline
    & HR & Monitoring setiap hari dilakukan pagi dan sore, dan minimal 1 jam. Untuk provisioning penambahan perangkat baru tentunya memakan waktu tergantung jumlah perangkat. Biasanya memakan waktu 2-3 hari tergantung banyaknya jumlah perangkat. \\
    \hline
    5 & TR & Bagaimana menurut Bapak efisiensi operasional dapat ditingkatkan dengan adanya otomasi? \\
    \hline
    & HR & Bicara otomatisasi tentunya dapat memakngkas banyak waktu, apalagi perangkat network sekarang banyak yang menggunakan feature “zero trust provisioning. Tantangan selanjutnya bukan hanya dari segi provisioning, namun juga dari sisi operations maintenance, apalagi kita maintain banyak perangkat jaringan sehingga log notifikasi menjadi sangat penting. Poin selanjutnya kita akan sangat terbantu dengan feature AI, yang mana dapat menghemat waktu troubleshooting jika ada sesuatu di jaringan. \\
    \hline
    6 & TR & Apakah ada kendala dalam kolaborasi antara tim jaringan dan tim lain, misalnya tim DevOps atau aplikasi, yang menurut Bapak bisa diatasi dengan otomatisasi? \\
    \hline
    & HR & Kendala kolaborasi dirasakan dengan tim aplikasi dimana akses ke aplikasi ini sering berubah dan dynamic, sehingga di sisi network perlu ada interfensi dan perubahan konfigurasi. \\
    \hline
    7 & TR & Bagaimana rencana Bapak untuk menangani pertumbuhan jaringan, baik dari sisi perangkat, kapasitas, maupun lokasi geografis? Apakah Bapak melihat otomatisasi sebagai solusi untuk itu? \\
    \hline
    & HR & Sebagai perusahaan corporate yang menangani banyak anak perusahaan, simplify the network device is the key, dan bantuan AI menjadi nilai lebih untuk membantu managed network operations. \\
    \hline
    8 & TR & Seberapa besar fleksibilitas yang Bapak butuhkan dari solusi network automation untuk bisa mengakomodasi perubahan jaringan di masa depan? \\
    \hline
    & HR & Dari 1-10 kebutuhan autimatsisasi ini bisa di kataka di poin 8. Artinya automatisasi dari sisi provisioning, monitoring dan troubleshooting dibutuhkan. \\
    \hline
    9 & TR & Apakah Bapak merasa proses manual saat ini dapat menyebabkan risiko keamanan atau ketidakpatuhan terhadap regulasi? \\
    \hline
    & HR & Jika dikatakan ketidakpatuhan terhadap regulasi tidak, namun memang dapat menyebabkan resiko di operations. \\
    \hline
    10 & TR & Bagaimana Bapak membayangkan network automation bisa membantu memperkuat keamanan jaringan PT. XYZ? \\
    \hline
    & HR & Bicara membantu memperkuat keaman jaringan tentunya dibarengi dengan optimisasi konfigurasi dan perangkat jaringan yang memadai. Tentunya ini akan sangan terbantu jika ada automatisasi. \\
    \hline
    11 & TR & Apakah ada kebutuhan untuk meningkatkan pengawasan dan pencatatan perubahan konfigurasi jaringan secara otomatis untuk keperluan audit atau keamanan? \\
    \hline
    & HR & Ya, saat ini perubahan konfigurasi masih manual menggunakan form. Jika dapat di digitalisasikan akan lebih baik. \\
    \hline
    12 & TR & Seberapa sering tim Bapak harus melakukan pemeliharaan rutin atau perbaikan yang bisa berpotensi terganggu oleh kesalahan manual? \\
    \hline
    & HR & Saat ini dibuat preventive maintenance pengecekan setiap 3 bulan. \\
    \hline
    13 & TR & Apakah Bapak sudah menggunakan atau merencanakan penggunaan alat monitoring jaringan yang mendukung otomatisasi untuk mendeteksi dan merespons masalah jaringan secara real-time? \\
    \hline
    & HR & Untuk automatisasi ada rencana kesana, namun masih dalam diskusi internal terkait implementasinya. \\
    \hline
    14 & TR & Apakah Bapak memiliki perangkat atau infrastruktur lama yang mungkin sulit diotomatisasi? Jika ya, bagaimana strategi Bapak untuk mengintegrasikan perangkat tersebut dengan solusi otomasi? \\
    \hline
    & HR & Ya banyak yang beum support automatisasi, biasanya akan dilakukan refreshment jika usia perangkat sudah 5 tahun. \\
    \hline
    15 & TR & Seberapa penting bagi Bapak untuk memiliki solusi network automation yang terintegrasi dengan sistem manajemen IT lainnya (misalnya, sistem monitoring, logging, atau ticketing)? \\
    \hline
    & HR & Di era digital dan Automatisasi saat ini, managed perangkat network dibantu dengan tools yang mumpuni menjadi sangat penting. \\
    \hline
    16 & TR & Bagaimana kesiapan tim Bapak untuk menerapkan network automation? Apakah ada kebutuhan untuk meningkatkan keterampilan mereka, misalnya dalam scripting, pemrograman, atau penggunaan API? \\
    \hline
    & HR & Jika pada masanya kita akan implement automation, maka tentunya capabilitas team juga akan di tingkatkan. Tapi saat ini kami focus refreshment perangkat agar bisa support automation. \\
    \hline
    17 & TR & Apakah menurut Bapak otomatisasi dapat mengubah cara kerja tim, seperti mempercepat waktu penyelesaian masalah atau mengurangi intervensi manual? \\
    \hline
    & HR & Betul, akan sangat mengubah cara kerja team menjadi lebih efisien, sehingga bisa focus pada strategi pengembangan jaringan disbanding stuck di troubleshooting issue. \\
    \hline
    18 & TR & Apa saja fitur utama yang Bapak cari dalam solusi network automation? Apakah Bapak lebih memprioritaskan aspek skalabilitas, fleksibilitas, atau keamanan? \\
    \hline
    & HR & Feature utama saat ini yang dibutuhkan adalah security, lalu lalu ke flexibilitas perangkat yang dimiliki saat ini, dimana banyak perangkat network yang memang masih memerlukan refreshment. \\
    \hline
    19 & TR & Seberapa penting bagi Bapak memiliki solusi yang mendukung multi-vendor dan protokol terbuka untuk menjaga fleksibilitas infrastruktur jaringan PT. XYZ? \\
    \hline
    & HR & Ini menjadi hal penting karena di kami banyak multibrand, sehingga flexibilitas tools menjadi sangat penting. \\
    \hline
\end{longtable}

\chapter*{Lampiran 2}

\begin{table}[h!]
    \centering
    \renewcommand{\arraystretch}{1.3}
    \begin{tabular}{|p{4cm}|p{10cm}|}
        \hline
        \textbf{Narasumber} & Haykal Rasidi \\
        \hline
        \textbf{Jabatan} & IT Infrastructure Manager PT. XYZ \\
        \hline
        \textbf{Tanggal} & 28 September 2024 \\
        \hline
        \textbf{Kode} & 
        \begin{tabular}{@{}ll@{}}
            HR & Haykal Rasidi \\
            TR & Tio Ramadhan \\
        \end{tabular} \\
        \hline
    \end{tabular}
\end{table}

\vspace{10pt}

\begin{longtable}{|p{4cm}|p{10cm}|}
    \hline
    \textbf{Topik} & \textbf{Isi Diskusi} \\
    \hline
    \endfirsthead

    \hline
    \textbf{Topik} & \textbf{Isi Diskusi} \\
    \hline
    \endhead

    \hline
    \multicolumn{2}{|r|}{{Lanjutan ke halaman berikutnya}} \\
    \hline
    \endfoot

    \hline
    \endlastfoot

    \textbf{Man Power} & \\
    \hline
    \textbf{TR} & Berapa jumlah tenaga kerja (man power) yang saat ini dibutuhkan untuk menjalankan operasi dan pemeliharaan infrastruktur jaringan secara manual? \\
    \textbf{HR} & Minimal 4 orang \\
    \hline
    \textbf{TR} & Apa saja kualifikasi dan keahlian teknis yang diperlukan dari tenaga kerja untuk mengelola infrastruktur jaringan tanpa otomatisasi? \\
    \textbf{HR} & Konfigurasi command line dan kemampuan engineer untuk melakukan troubleshooting jaringan. \\
    \hline
    \textbf{TR} & Berapa jumlah tim yang terlibat dalam proses troubleshooting jaringan secara manual? \\
    \textbf{HR} & Tergantung seberapa besar isu-nya; lebih sering 1 orang per isu, jika ada kendala bisa melibatkan 2–3 orang. \\
    \hline
    \textbf{TR} & Apakah jumlah tenaga kerja yang diperlukan dapat dikurangi jika network automation diterapkan? Jika ya, berapa perkiraan pengurangannya? \\
    \textbf{HR} & Kemungkinan iya, karena kompleksitas berkurang maka tenaga diperkirakan akan berkurang 30-50\%. \\
    \hline
    \textbf{TR} & Apakah tenaga kerja yang ada saat ini memiliki keterampilan untuk mengelola sistem jaringan otomatis, atau diperlukan pelatihan tambahan? \\
    \textbf{HR} & Ada, tapi perlu peningkatan skill dan pelatihan tambahan. \\
    \hline

    \textbf{Man Cost} & \\
    \hline
    \textbf{TR} & Berapa besar biaya yang dihabiskan untuk tenaga kerja (man cost) dalam mengelola jaringan tanpa otomatisasi? (Ini mencakup gaji, tunjangan, dan biaya pelatihan) \\
    \textbf{HR} & Confidential \\
    \hline
    \textbf{TR} & Apakah ada biaya tambahan yang harus dikeluarkan untuk overtime atau tenaga kerja tambahan selama ada gangguan jaringan besar? \\
    \textbf{HR} & Di environment kami tidak ada biaya tambahan karena sistem overtime tidak berlaku. \\
    \hline
    \textbf{TR} & Bagaimana perbandingan biaya tenaga kerja untuk mengelola jaringan manual vs otomatis dalam hal penghematan biaya operasional? \\
    \textbf{HR} & Diperkirakan penghematan bisa berkisar 30–50\%. \\
    \hline
    \textbf{TR} & Apakah ada penghematan dalam biaya perekrutan atau outsourcing jika network automation diterapkan? \\
    \textbf{HR} & Iya, karena tenaga kerja akan berkurang. \\
    \hline
    \textbf{TR} & Berapa biaya yang dianggarkan untuk pelatihan tenaga kerja dalam mengoperasikan solusi otomatisasi jaringan? \\
    \textbf{HR} & Confidential. \\
    \hline

    \textbf{Man Hours} & \\
    \hline
    \textbf{TR} & Berapa jumlah jam kerja yang dihabiskan setiap bulan untuk melakukan konfigurasi, pemeliharaan, dan troubleshooting jaringan secara manual? \\
    \textbf{HR} & Asumsi per minggu 40 jam kerja; 1 bulan 160 jam kerja, untuk pemeliharaan sendiri menghabiskan waktu kurang lebih 60\%-70\%. \\
    \hline
    \textbf{TR} & Berapa rata-rata waktu yang dihabiskan untuk menyelesaikan masalah jaringan secara manual (troubleshooting time) vs waktu yang diperlukan jika menggunakan network automation? \\
    \textbf{HR} & Untuk manual konfigurasi tergantung masalahnya, rata-rata 1–3 jam. Dengan automation diperkirakan bisa memangkas waktu ½ atau lebih. \\
    \hline
    \textbf{TR} & Berapa lama waktu yang dibutuhkan untuk men-deploy perubahan jaringan (misalnya update, upgrade, atau penambahan perangkat) secara manual? \\
    \textbf{HR} & Konfigurasi manual biasanya dilakukan dalam 1–2 jam. \\
    \hline
    \textbf{TR} & Apakah network automation dapat mengurangi waktu implementasi dan maintenance? Jika ya, berapa perkiraan penghematannya dalam hitungan man hours? \\
    \textbf{HR} & Jika manual, 1 switch dikonfigurasi dalam 1–2 jam; dengan automation hanya memakan waktu 15 menit. \\
    \hline
    \textbf{TR} & Bagaimana network automation dapat mempercepat proses deteksi dan resolusi insiden jaringan, dibandingkan dengan man hours yang digunakan pada pengelolaan manual? \\
    \textbf{HR} & Network Automation akan 80\% lebih cepat dibandingkan manual monitoring, terutama dengan bantuan Artificial Intelligence yang memberikan panduan troubleshooting. \\
    \hline

\end{longtable}
